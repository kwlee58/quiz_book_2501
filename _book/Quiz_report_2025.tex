% Options for packages loaded elsewhere
\PassOptionsToPackage{unicode}{hyperref}
\PassOptionsToPackage{hyphens}{url}
%
\documentclass[
]{book}
\usepackage{amsmath,amssymb}
\usepackage{iftex}
\ifPDFTeX
  \usepackage[T1]{fontenc}
  \usepackage[utf8]{inputenc}
  \usepackage{textcomp} % provide euro and other symbols
\else % if luatex or xetex
  \usepackage{unicode-math} % this also loads fontspec
  \defaultfontfeatures{Scale=MatchLowercase}
  \defaultfontfeatures[\rmfamily]{Ligatures=TeX,Scale=1}
\fi
\usepackage{lmodern}
\ifPDFTeX\else
  % xetex/luatex font selection
\fi
% Use upquote if available, for straight quotes in verbatim environments
\IfFileExists{upquote.sty}{\usepackage{upquote}}{}
\IfFileExists{microtype.sty}{% use microtype if available
  \usepackage[]{microtype}
  \UseMicrotypeSet[protrusion]{basicmath} % disable protrusion for tt fonts
}{}
\makeatletter
\@ifundefined{KOMAClassName}{% if non-KOMA class
  \IfFileExists{parskip.sty}{%
    \usepackage{parskip}
  }{% else
    \setlength{\parindent}{0pt}
    \setlength{\parskip}{6pt plus 2pt minus 1pt}}
}{% if KOMA class
  \KOMAoptions{parskip=half}}
\makeatother
\usepackage{xcolor}
\usepackage{longtable,booktabs,array}
\usepackage{calc} % for calculating minipage widths
% Correct order of tables after \paragraph or \subparagraph
\usepackage{etoolbox}
\makeatletter
\patchcmd\longtable{\par}{\if@noskipsec\mbox{}\fi\par}{}{}
\makeatother
% Allow footnotes in longtable head/foot
\IfFileExists{footnotehyper.sty}{\usepackage{footnotehyper}}{\usepackage{footnote}}
\makesavenoteenv{longtable}
\usepackage{graphicx}
\makeatletter
\def\maxwidth{\ifdim\Gin@nat@width>\linewidth\linewidth\else\Gin@nat@width\fi}
\def\maxheight{\ifdim\Gin@nat@height>\textheight\textheight\else\Gin@nat@height\fi}
\makeatother
% Scale images if necessary, so that they will not overflow the page
% margins by default, and it is still possible to overwrite the defaults
% using explicit options in \includegraphics[width, height, ...]{}
\setkeys{Gin}{width=\maxwidth,height=\maxheight,keepaspectratio}
% Set default figure placement to htbp
\makeatletter
\def\fps@figure{htbp}
\makeatother
\setlength{\emergencystretch}{3em} % prevent overfull lines
\providecommand{\tightlist}{%
  \setlength{\itemsep}{0pt}\setlength{\parskip}{0pt}}
\setcounter{secnumdepth}{5}
\usepackage{fontspec}
\setmainfont{KoPubWorldDotum}
\ifLuaTeX
  \usepackage{selnolig}  % disable illegal ligatures
\fi
\usepackage{bookmark}
\IfFileExists{xurl.sty}{\usepackage{xurl}}{} % add URL line breaks if available
\urlstyle{same}
\hypersetup{
  pdftitle={Quiz Reports: 2025 Spring},
  pdfauthor={이기원 (Kee-Won Lee)},
  hidelinks,
  pdfcreator={LaTeX via pandoc}}

\title{Quiz Reports: 2025 Spring}
\author{이기원 (Kee-Won Lee)}
\date{2025-04-19}

\begin{document}
\maketitle

{
\setcounter{tocdepth}{1}
\tableofcontents
}
\chapter{소개}\label{uxc18cuxac1c}

이 문서는 2025년 봄학기 동안 수행된 주요 퀴즈 결과 및 분석을 정리한 보고서입니다.

각 장에서는 개별 퀴즈의 응답 분포, 정답률, 인지편향 관찰 결과 등을 시각화하고 요약합니다.

분석 대상은 다음과 같습니다:

\begin{itemize}
\tightlist
\item
  Wason Selection Task 정답률 변화
\item
  프레이밍 효과에 따른 반응 차이
\item
  Oxford Happiness Questionnaire 응답 패턴
\item
  언어별 응답지 차이 (한글 vs 영어)
\end{itemize}

R과 R Markdown을 기반으로 자동화된 시각화 및 분석을 포함합니다.

\begin{center}\rule{0.5\linewidth}{0.5pt}\end{center}

\chapter{1주차 데이터 실험 집계}\label{uxc8fcuxcc28-uxb370uxc774uxd130-uxc2e4uxd5d8-uxc9d1uxacc4}

Placeholder

\section{실험의 목적}\label{uxc2e4uxd5d8uxc758-uxbaa9uxc801}

\subsection{Red, Black을 잘못 표시한 사람들}\label{red-blackuxc744-uxc798uxbabb-uxd45cuxc2dcuxd55c-uxc0acuxb78cuxb4e4}

\subsection{응답인원의 Red, Black}\label{uxc751uxb2f5uxc778uxc6d0uxc758-red-black}

\section{Q1. Dewey as good as elected, statistics convince Roper}\label{q1.-dewey-as-good-as-elected-statistics-convince-roper}

\subsection{Roper(Counts)}\label{ropercounts}

\subsection{Roper(\%)}\label{roper}

\section{Q2. Statistics is the science of learning from data, \ldots{}}\label{q2.-statistics-is-the-science-of-learning-from-data}

\subsection{ASA(Counts)}\label{asacounts}

\subsection{ASA(\%)}\label{asa}

\section{Q3. How to lie with statistics}\label{q3.-how-to-lie-with-statistics}

\subsection{D.Huff(Counts)}\label{d.huffcounts}

\subsection{D.Huff(\%)}\label{d.huff}

\section{Q4. 종부세}\label{q4.-uxc885uxbd80uxc138}

\subsection{질문지 선택지에 부연설명}\label{uxc9c8uxbb38uxc9c0-uxc120uxd0dduxc9c0uxc5d0-uxbd80uxc5f0uxc124uxba85}

\subsection{집계}\label{uxc9d1uxacc4}

\subsection{\% 비교.}\label{uxbe44uxad50.}

\subsection{Mosaic Plot}\label{mosaic-plot}

\subsection{\% 합계}\label{uxd569uxacc4}

\section{마감 시간으로부터 제출 시간의 분포}\label{uxb9c8uxac10-uxc2dcuxac04uxc73cuxb85cuxbd80uxd130-uxc81cuxcd9c-uxc2dcuxac04uxc758-uxbd84uxd3ec}

\subsection{분포표}\label{uxbd84uxd3ecuxd45c}

\subsection{날마다 고르게 제출하는가?}\label{uxb0a0uxb9c8uxb2e4-uxace0uxb974uxac8c-uxc81cuxcd9cuxd558uxb294uxac00}

\subsection{막대그래프}\label{uxb9c9uxb300uxadf8uxb798uxd504}

\subsection{Red, Black 간에 닮았는가?}\label{red-black-uxac04uxc5d0-uxb2eeuxc558uxb294uxac00}

\subsection{Mosaic Plot}\label{mosaic-plot-1}

\chapter{2주차 데이터 실험 집계}\label{uxc8fcuxcc28-uxb370uxc774uxd130-uxc2e4uxd5d8-uxc9d1uxacc4-1}

Placeholder

\section{실험의 목적}\label{uxc2e4uxd5d8uxc758-uxbaa9uxc801-1}

\subsection{Red, Black을 잘못 표시한 사람들}\label{red-blackuxc744-uxc798uxbabb-uxd45cuxc2dcuxd55c-uxc0acuxb78cuxb4e4-1}

\subsection{응답인원의 Red, Black}\label{uxc751uxb2f5uxc778uxc6d0uxc758-red-black-1}

\section{Q1. 춘추전국시대에 국가통계관리의 중요성 강조}\label{q1.-uxcd98uxcd94uxc804uxad6duxc2dcuxb300uxc5d0-uxad6duxac00uxd1b5uxacc4uxad00uxb9acuxc758-uxc911uxc694uxc131-uxac15uxc870}

\subsection{관자(집계표)}\label{uxad00uxc790uxc9d1uxacc4uxd45c}

\subsection{관자(\%)}\label{uxad00uxc790}

\section{Q2. 국가정책을 수립하는 데 통계의 역할}\label{q2.-uxad6duxac00uxc815uxcc45uxc744-uxc218uxb9bduxd558uxb294-uxb370-uxd1b5uxacc4uxc758-uxc5eduxd560}

\subsection{통계의 중요성(집계표)}\label{uxd1b5uxacc4uxc758-uxc911uxc694uxc131uxc9d1uxacc4uxd45c}

\subsection{통계의 중요성(\%)}\label{uxd1b5uxacc4uxc758-uxc911uxc694uxc131}

\section{Q3. 우리나라 생산가능인구 감소 시기}\label{q3.-uxc6b0uxb9acuxb098uxb77c-uxc0dduxc0b0uxac00uxb2a5uxc778uxad6c-uxac10uxc18c-uxc2dcuxae30}

\subsection{생산가능인구 감소 시기(집계표)}\label{uxc0dduxc0b0uxac00uxb2a5uxc778uxad6c-uxac10uxc18c-uxc2dcuxae30uxc9d1uxacc4uxd45c}

\subsection{생산가능인구 감소 시기(\%)}\label{uxc0dduxc0b0uxac00uxb2a5uxc778uxad6c-uxac10uxc18c-uxc2dcuxae30}

\section{Q4. 우리나라 총인구 최대 시기}\label{q4.-uxc6b0uxb9acuxb098uxb77c-uxcd1duxc778uxad6c-uxcd5cuxb300-uxc2dcuxae30}

\subsection{총인구 최대 시기(집계표)}\label{uxcd1duxc778uxad6c-uxcd5cuxb300-uxc2dcuxae30uxc9d1uxacc4uxd45c}

\subsection{총인구 최대 시기(\%)}\label{uxcd1duxc778uxad6c-uxcd5cuxb300-uxc2dcuxae30}

\section{Q5. 소멸위험 단계 개선 지역}\label{q5.-uxc18cuxba78uxc704uxd5d8-uxb2e8uxacc4-uxac1cuxc120-uxc9c0uxc5ed}

\subsection{소멸위험 단계 개선 지역(집계표)}\label{uxc18cuxba78uxc704uxd5d8-uxb2e8uxacc4-uxac1cuxc120-uxc9c0uxc5eduxc9d1uxacc4uxd45c}

\subsection{소멸위험 단계 개선 지역(\%)}\label{uxc18cuxba78uxc704uxd5d8-uxb2e8uxacc4-uxac1cuxc120-uxc9c0uxc5ed}

\section{Q6. 조출생률과 합계출산율}\label{q6.-uxc870uxcd9cuxc0dduxb960uxacfc-uxd569uxacc4uxcd9cuxc0b0uxc728}

\subsection{조출생률과 합계출산율(집계표)}\label{uxc870uxcd9cuxc0dduxb960uxacfc-uxd569uxacc4uxcd9cuxc0b0uxc728uxc9d1uxacc4uxd45c}

\subsection{조출생률과 합계출산율(\%)}\label{uxc870uxcd9cuxc0dduxb960uxacfc-uxd569uxacc4uxcd9cuxc0b0uxc728}

\section{Q7. 눈속임 그래프(Cheating Charts)}\label{q7.-uxb208uxc18duxc784-uxadf8uxb798uxd504cheating-charts}

\subsection{집계표}\label{uxc9d1uxacc4uxd45c}

\subsection{\% 비교}\label{uxbe44uxad50}

\subsection{집계}\label{uxc9d1uxacc4-1}

\subsection{\%}\label{section}

\subsection{Mosaic Plot}\label{mosaic-plot-2}

\section{마감 시간으로부터 제출 시간의 분포}\label{uxb9c8uxac10-uxc2dcuxac04uxc73cuxb85cuxbd80uxd130-uxc81cuxcd9c-uxc2dcuxac04uxc758-uxbd84uxd3ec-1}

\subsection{분포표}\label{uxbd84uxd3ecuxd45c-1}

\subsection{날마다 고르게 제출하는가?}\label{uxb0a0uxb9c8uxb2e4-uxace0uxb974uxac8c-uxc81cuxcd9cuxd558uxb294uxac00-1}

\subsection{막대그래프}\label{uxb9c9uxb300uxadf8uxb798uxd504-1}

\subsection{Red, Black 간에 닮았는가?}\label{red-black-uxac04uxc5d0-uxb2eeuxc558uxb294uxac00-1}

\subsection{Mosaic Plot}\label{mosaic-plot-3}

\chapter{3주차 데이터 실험 집계}\label{uxc8fcuxcc28-uxb370uxc774uxd130-uxc2e4uxd5d8-uxc9d1uxacc4-2}

Placeholder

\section{실험의 목적}\label{uxc2e4uxd5d8uxc758-uxbaa9uxc801-2}

\subsection{Red, Black을 잘못 표시한 사람들}\label{red-blackuxc744-uxc798uxbabb-uxd45cuxc2dcuxd55c-uxc0acuxb78cuxb4e4-2}

\subsection{응답인원의 Red, Black}\label{uxc751uxb2f5uxc778uxc6d0uxc758-red-black-2}

\section{Q1. 국세와 지방세 비중}\label{q1.-uxad6duxc138uxc640-uxc9c0uxbc29uxc138-uxbe44uxc911}

\subsection{국세와 지방세 비중(집계표)}\label{uxad6duxc138uxc640-uxc9c0uxbc29uxc138-uxbe44uxc911uxc9d1uxacc4uxd45c}

\subsection{국세와 지방세 비중(\%)}\label{uxad6duxc138uxc640-uxc9c0uxbc29uxc138-uxbe44uxc911}

\section{Q2. 조세부담률}\label{q2.-uxc870uxc138uxbd80uxb2f4uxb960}

\subsection{조세부담률(집계표)}\label{uxc870uxc138uxbd80uxb2f4uxb960uxc9d1uxacc4uxd45c}

\subsection{조세부담률(\%)}\label{uxc870uxc138uxbd80uxb2f4uxb960}

\section{Q3. OECD 국민부담률}\label{q3.-oecd-uxad6duxbbfcuxbd80uxb2f4uxb960}

\subsection{OECD 국민부담률(집계표)}\label{oecd-uxad6duxbbfcuxbd80uxb2f4uxb960uxc9d1uxacc4uxd45c}

\subsection{OECD 국민부담률(\%)}\label{oecd-uxad6duxbbfcuxbd80uxb2f4uxb960}

\section{Q4. 과세대상 근로소득 1,200만 원}\label{q4.-uxacfcuxc138uxb300uxc0c1-uxadfcuxb85cuxc18cuxb4dd-1200uxb9cc-uxc6d0}

\subsection{과세대상 근로소득 1,200만 원(집계표)}\label{uxacfcuxc138uxb300uxc0c1-uxadfcuxb85cuxc18cuxb4dd-1200uxb9cc-uxc6d0uxc9d1uxacc4uxd45c}

\subsection{과세대상 근로소득 1,200만 원(\%)}\label{uxacfcuxc138uxb300uxc0c1-uxadfcuxb85cuxc18cuxb4dd-1200uxb9cc-uxc6d0}

\section{Q5. 소득세 실효세율}\label{q5.-uxc18cuxb4dduxc138-uxc2e4uxd6a8uxc138uxc728}

\subsection{소득세 실효세율(집계표)}\label{uxc18cuxb4dduxc138-uxc2e4uxd6a8uxc138uxc728uxc9d1uxacc4uxd45c}

\subsection{소득세 실효세율(\%)}\label{uxc18cuxb4dduxc138-uxc2e4uxd6a8uxc138uxc728}

\section{Q6. 기업규모별 과세 현황}\label{q6.-uxae30uxc5c5uxaddcuxbaa8uxbcc4-uxacfcuxc138-uxd604uxd669}

\subsection{기업규모별 과세 현황(집계표)}\label{uxae30uxc5c5uxaddcuxbaa8uxbcc4-uxacfcuxc138-uxd604uxd669uxc9d1uxacc4uxd45c}

\subsection{기업규모별 과세 현황(\%)}\label{uxae30uxc5c5uxaddcuxbaa8uxbcc4-uxacfcuxc138-uxd604uxd669}

\section{Q7. 국민부담률 적정 수준 : 아일랜드와 OECD 평균}\label{q7.-uxad6duxbbfcuxbd80uxb2f4uxb960-uxc801uxc815-uxc218uxc900-uxc544uxc77cuxb79cuxb4dcuxc640-oecd-uxd3c9uxade0}

\subsection{집계표}\label{uxc9d1uxacc4uxd45c-1}

\subsection{\% 비교}\label{uxbe44uxad50-1}

\subsection{\% 합계}\label{uxd569uxacc4-1}

\subsection{Mosaic Plot}\label{mosaic-plot-4}

\section{마감 시간으로부터 제출 시간의 분포}\label{uxb9c8uxac10-uxc2dcuxac04uxc73cuxb85cuxbd80uxd130-uxc81cuxcd9c-uxc2dcuxac04uxc758-uxbd84uxd3ec-2}

\subsection{분포표}\label{uxbd84uxd3ecuxd45c-2}

\subsection{날마다 고르게 제출하는가?}\label{uxb0a0uxb9c8uxb2e4-uxace0uxb974uxac8c-uxc81cuxcd9cuxd558uxb294uxac00-2}

\subsection{막대그래프}\label{uxb9c9uxb300uxadf8uxb798uxd504-2}

\subsection{Red, Black 간에 닮았는가?}\label{red-black-uxac04uxc5d0-uxb2eeuxc558uxb294uxac00-2}

\subsection{Mosaic Plot}\label{mosaic-plot-5}

\chapter{4주차 데이터 실험 집계}\label{uxc8fcuxcc28-uxb370uxc774uxd130-uxc2e4uxd5d8-uxc9d1uxacc4-3}

Placeholder

\section{실험의 목적}\label{uxc2e4uxd5d8uxc758-uxbaa9uxc801-3}

\subsection{Red, Black을 잘못 표시한 사람들}\label{red-blackuxc744-uxc798uxbabb-uxd45cuxc2dcuxd55c-uxc0acuxb78cuxb4e4-3}

\subsection{응답인원의 Red, Black}\label{uxc751uxb2f5uxc778uxc6d0uxc758-red-black-3}

\section{Q1. 세종대왕 시대 조세제도}\label{q1.-uxc138uxc885uxb300uxc655-uxc2dcuxb300-uxc870uxc138uxc81cuxb3c4}

\subsection{조선초기 조세제도}\label{uxc870uxc120uxcd08uxae30-uxc870uxc138uxc81cuxb3c4}

\subsection{조선초기 조세제도(\%)}\label{uxc870uxc120uxcd08uxae30-uxc870uxc138uxc81cuxb3c4-1}

\section{Q2. 공법도입에 대한 대신들의 찬성율}\label{q2.-uxacf5uxbc95uxb3c4uxc785uxc5d0-uxb300uxd55c-uxb300uxc2e0uxb4e4uxc758-uxcc2cuxc131uxc728}

\subsection{공법도입과 대신들(집계표)}\label{uxacf5uxbc95uxb3c4uxc785uxacfc-uxb300uxc2e0uxb4e4uxc9d1uxacc4uxd45c}

\subsection{공법도입과 대신들(\%)}\label{uxacf5uxbc95uxb3c4uxc785uxacfc-uxb300uxc2e0uxb4e4}

\section{Q3. 공법도입과 품관촌민들의 찬반}\label{q3.-uxacf5uxbc95uxb3c4uxc785uxacfc-uxd488uxad00uxcd0cuxbbfcuxb4e4uxc758-uxcc2cuxbc18}

\subsection{품관촌민들의 찬반(집계표)}\label{uxd488uxad00uxcd0cuxbbfcuxb4e4uxc758-uxcc2cuxbc18uxc9d1uxacc4uxd45c}

\subsection{품관촌민들의 찬반(\%)}\label{uxd488uxad00uxcd0cuxbbfcuxb4e4uxc758-uxcc2cuxbc18}

\section{Q4. 공법}\label{q4.-uxacf5uxbc95}

\subsection{기본세율}\label{uxae30uxbcf8uxc138uxc728}

\subsection{기본세율(\%)}\label{uxae30uxbcf8uxc138uxc728-1}

\section{Q5. 1423년 조선시대 호구와 인구}\label{q5.-1423uxb144-uxc870uxc120uxc2dcuxb300-uxd638uxad6cuxc640-uxc778uxad6c}

\subsection{호구와 인구}\label{uxd638uxad6cuxc640-uxc778uxad6c}

\subsection{호구와 인구(\%)}\label{uxd638uxad6cuxc640-uxc778uxad6c-1}

\section{Q6. 지방관료와 품관촌민}\label{q6.-uxc9c0uxbc29uxad00uxb8ccuxc640-uxd488uxad00uxcd0cuxbbfc}

\subsection{찬반이 반대인 곳(집계표)}\label{uxcc2cuxbc18uxc774-uxbc18uxb300uxc778-uxacf3uxc9d1uxacc4uxd45c}

\subsection{찬반이 반대인 곳(\%)}\label{uxcc2cuxbc18uxc774-uxbc18uxb300uxc778-uxacf3}

\section{Q7. 부연설명의 효과 : 주당 근로 69시간제 도입 찬반}\label{q7.-uxbd80uxc5f0uxc124uxba85uxc758-uxd6a8uxacfc-uxc8fcuxb2f9-uxadfcuxb85c-69uxc2dcuxac04uxc81c-uxb3c4uxc785-uxcc2cuxbc18}

\subsection{집계}\label{uxc9d1uxacc4-2}

\subsection{\% 비교}\label{uxbe44uxad50-2}

\subsection{\% 합계}\label{uxd569uxacc4-2}

\subsection{Mosaic Plot}\label{mosaic-plot-6}

\section{마감 시간으로부터 제출 시간의 분포}\label{uxb9c8uxac10-uxc2dcuxac04uxc73cuxb85cuxbd80uxd130-uxc81cuxcd9c-uxc2dcuxac04uxc758-uxbd84uxd3ec-3}

\subsection{분포표}\label{uxbd84uxd3ecuxd45c-3}

\subsection{날마다 고르게 제출하는가?}\label{uxb0a0uxb9c8uxb2e4-uxace0uxb974uxac8c-uxc81cuxcd9cuxd558uxb294uxac00-3}

\subsection{막대그래프}\label{uxb9c9uxb300uxadf8uxb798uxd504-3}

\subsection{Red, Black 간에 닮았는가?}\label{red-black-uxac04uxc5d0-uxb2eeuxc558uxb294uxac00-3}

\subsection{Mosaic Plot}\label{mosaic-plot-7}

\chapter{5주차 데이터 실험 집계}\label{uxc8fcuxcc28-uxb370uxc774uxd130-uxc2e4uxd5d8-uxc9d1uxacc4-4}

Placeholder

\section{실험의 목적}\label{uxc2e4uxd5d8uxc758-uxbaa9uxc801-4}

\subsection{Red, Black을 잘못 표시한 사람들}\label{red-blackuxc744-uxc798uxbabb-uxd45cuxc2dcuxd55c-uxc0acuxb78cuxb4e4-4}

\subsection{응답인원의 Red, Black}\label{uxc751uxb2f5uxc778uxc6d0uxc758-red-black-4}

\section{Q1. 한글의 문자 유형}\label{q1.-uxd55cuxae00uxc758-uxbb38uxc790-uxc720uxd615}

\subsection{한글은 민주 문자}\label{uxd55cuxae00uxc740-uxbbfcuxc8fc-uxbb38uxc790}

\subsection{한글은 민주 문자(\%)}\label{uxd55cuxae00uxc740-uxbbfcuxc8fc-uxbb38uxc790-1}

\section{Q2. 정보혁명과 문자 체계}\label{q2.-uxc815uxbcf4uxd601uxba85uxacfc-uxbb38uxc790-uxccb4uxacc4}

\subsection{정보혁명을 이끄는 문자는 한글(집계표)}\label{uxc815uxbcf4uxd601uxba85uxc744-uxc774uxb044uxb294-uxbb38uxc790uxb294-uxd55cuxae00uxc9d1uxacc4uxd45c}

\subsection{정보혁명을 이끄는 문자는 한글(\%)}\label{uxc815uxbcf4uxd601uxba85uxc744-uxc774uxb044uxb294-uxbb38uxc790uxb294-uxd55cuxae00}

\section{Q3. 알기 힘든 전문 용어}\label{q3.-uxc54cuxae30-uxd798uxb4e0-uxc804uxbb38-uxc6a9uxc5b4}

\subsection{몇 개나 아나요?(집계표)}\label{uxba87-uxac1cuxb098-uxc544uxb098uxc694uxc9d1uxacc4uxd45c}

\subsection{몇 개나 아나요?(\%)}\label{uxba87-uxac1cuxb098-uxc544uxb098uxc694}

\section{Q4. 해방직후 비문해율}\label{q4.-uxd574uxbc29uxc9c1uxd6c4-uxbe44uxbb38uxd574uxc728}

\subsection{집계}\label{uxc9d1uxacc4-3}

\subsection{\%}\label{section-1}

\section{Q5. 세대간 문해력 격차}\label{q5.-uxc138uxb300uxac04-uxbb38uxd574uxb825-uxaca9uxcc28}

\subsection{집계}\label{uxc9d1uxacc4-4}

\subsection{\%}\label{section-2}

\section{Q6. 문해력 격차의 파급효과}\label{q6.-uxbb38uxd574uxb825-uxaca9uxcc28uxc758-uxd30cuxae09uxd6a8uxacfc}

\subsection{집계}\label{uxc9d1uxacc4-5}

\subsection{\%}\label{section-3}

\section{Q7. 프레임을 설정하는 단어의 힘}\label{q7.-uxd504uxb808uxc784uxc744-uxc124uxc815uxd558uxb294-uxb2e8uxc5b4uxc758-uxd798}

\subsection{집계}\label{uxc9d1uxacc4-6}

\subsection{\% 비교.}\label{uxbe44uxad50.-1}

\subsection{\% 합산}\label{uxd569uxc0b0}

\subsection{Mosaic Plot}\label{mosaic-plot-8}

\section{마감 시간으로부터 제출 시간의 분포}\label{uxb9c8uxac10-uxc2dcuxac04uxc73cuxb85cuxbd80uxd130-uxc81cuxcd9c-uxc2dcuxac04uxc758-uxbd84uxd3ec-4}

\subsection{분포표}\label{uxbd84uxd3ecuxd45c-4}

\subsection{날마다 고르게 제출하는가?}\label{uxb0a0uxb9c8uxb2e4-uxace0uxb974uxac8c-uxc81cuxcd9cuxd558uxb294uxac00-4}

\subsection{막대그래프}\label{uxb9c9uxb300uxadf8uxb798uxd504-4}

\subsection{Red, Black 간에 닮았는가?}\label{red-black-uxac04uxc5d0-uxb2eeuxc558uxb294uxac00-4}

\subsection{Mosaic Plot}\label{mosaic-plot-9}

\chapter{국민문해력조사 집계 결과}\label{uxad6duxbbfcuxbb38uxd574uxb825uxc870uxc0ac-uxc9d1uxacc4-uxacb0uxacfc}

Placeholder

\subsection{\texorpdfstring{\texttt{factor} 변환}{factor 변환}}\label{factor-uxbcc0uxd658}

\section{응답 집계}\label{uxc751uxb2f5-uxc9d1uxacc4}

\section{막대그래프}\label{uxb9c9uxb300uxadf8uxb798uxd504-5}

\subsection{\texorpdfstring{\texttt{barplot}}{barplot}}\label{barplot}

\subsection{ggplot}\label{ggplot}

\section{문해력 점수 계산}\label{uxbb38uxd574uxb825-uxc810uxc218-uxacc4uxc0b0}

\subsection{정답과 대조하여 R(Right)/W(Wrong) 표시}\label{uxc815uxb2f5uxacfc-uxb300uxc870uxd558uxc5ec-rrightwwrong-uxd45cuxc2dc}

\subsection{학생별 점수 산출}\label{uxd559uxc0dduxbcc4-uxc810uxc218-uxc0b0uxcd9c}

\section{Red and Black 비교}\label{red-and-black-uxbe44uxad50}

\subsection{Summary}\label{summary}

\subsection{줄기-잎 그림}\label{uxc904uxae30-uxc78e-uxadf8uxb9bc}

\subsection{Box Plots}\label{box-plots}

\subsection{QQ plot}\label{qq-plot}

\subsection{ECDF plot}\label{ecdf-plot}

\subsection{t-test}\label{t-test}

\section{문해력 등급 판정}\label{uxbb38uxd574uxb825-uxb4f1uxae09-uxd310uxc815}

\subsection{분포표}\label{uxbd84uxd3ecuxd45c-5}

\subsection{Red and Black}\label{red-and-black}

\section{유형별 정답률}\label{uxc720uxd615uxbcc4-uxc815uxb2f5uxb960}

\section{어려운 문제?}\label{uxc5b4uxb824uxc6b4-uxbb38uxc81c}

\subsection{정답률 80\% 이하}\label{uxc815uxb2f5uxb960-80-uxc774uxd558}

\subsection{정답률 70\% 이하}\label{uxc815uxb2f5uxb960-70-uxc774uxd558}

\subsection{정답률 60\% 이하}\label{uxc815uxb2f5uxb960-60-uxc774uxd558}

\subsection{정답률 50\% 이하}\label{uxc815uxb2f5uxb960-50-uxc774uxd558}

\section{정답률이 낮은 문제들}\label{uxc815uxb2f5uxb960uxc774-uxb0aeuxc740-uxbb38uxc81cuxb4e4}

\subsection{문6.}\label{uxbb386.}

\subsection{문9.}\label{uxbb389.}

\subsection{문12.}\label{uxbb3812.}

\subsection{문15.}\label{uxbb3815.}

\subsection{문17.}\label{uxbb3817.}

\subsection{문22.}\label{uxbb3822.}

\chapter{6주차 데이터 실험 집계}\label{uxc8fcuxcc28-uxb370uxc774uxd130-uxc2e4uxd5d8-uxc9d1uxacc4-5}

Placeholder

\section{실험의 목적}\label{uxc2e4uxd5d8uxc758-uxbaa9uxc801-5}

\subsection{Red, Black을 잘못 표시한 사람들}\label{red-blackuxc744-uxc798uxbabb-uxd45cuxc2dcuxd55c-uxc0acuxb78cuxb4e4-5}

\subsection{응답인원의 Red, Black}\label{uxc751uxb2f5uxc778uxc6d0uxc758-red-black-5}

\section{Q1. 월간 독서율}\label{q1.-uxc6d4uxac04-uxb3c5uxc11cuxc728}

\subsection{집계}\label{uxc9d1uxacc4-7}

\subsection{\%}\label{section-4}

\section{Q2. 지역 및 지역크기별 가구수 비례 무작위추출법}\label{q2.-uxc9c0uxc5ed-uxbc0f-uxc9c0uxc5eduxd06cuxae30uxbcc4-uxac00uxad6cuxc218-uxbe44uxb840-uxbb34uxc791uxc704uxcd94uxcd9cuxbc95}

\subsection{집계}\label{uxc9d1uxacc4-8}

\subsection{\%}\label{section-5}

\section{Q3. 한달 독서량의 분포}\label{q3.-uxd55cuxb2ec-uxb3c5uxc11cuxb7c9uxc758-uxbd84uxd3ec}

\subsection{집계}\label{uxc9d1uxacc4-9}

\subsection{\%}\label{section-6}

\section{Q4. 최근 1개월간 독서량}\label{q4.-uxcd5cuxadfc-1uxac1cuxc6d4uxac04-uxb3c5uxc11cuxb7c9}

\subsection{집계}\label{uxc9d1uxacc4-10}

\subsection{\%}\label{section-7}

\section{Q5. 20대의 연간독서율}\label{q5.-20uxb300uxc758-uxc5f0uxac04uxb3c5uxc11cuxc728}

\subsection{집계}\label{uxc9d1uxacc4-11}

\subsection{\%}\label{section-8}

\section{Q6. 50대의 연간독서율}\label{q6.-50uxb300uxc758-uxc5f0uxac04uxb3c5uxc11cuxc728}

\subsection{집계}\label{uxc9d1uxacc4-12}

\subsection{\%}\label{section-9}

\section{Q7. The more, the better? : 내가 남보다, 혹은 남이 나보다}\label{q7.-the-more-the-better-uxb0b4uxac00-uxb0a8uxbcf4uxb2e4-uxd639uxc740-uxb0a8uxc774-uxb098uxbcf4uxb2e4}

\subsection{집계}\label{uxc9d1uxacc4-13}

\subsection{\% 비교.}\label{uxbe44uxad50.-2}

\subsection{합산(\%)}\label{uxd569uxc0b0-1}

\subsection{Mosaic Plot}\label{mosaic-plot-10}

\section{마감 시간으로부터 제출 시간의 분포}\label{uxb9c8uxac10-uxc2dcuxac04uxc73cuxb85cuxbd80uxd130-uxc81cuxcd9c-uxc2dcuxac04uxc758-uxbd84uxd3ec-5}

\subsection{분포표}\label{uxbd84uxd3ecuxd45c-6}

\subsection{날마다 고르게 제출하는가?}\label{uxb0a0uxb9c8uxb2e4-uxace0uxb974uxac8c-uxc81cuxcd9cuxd558uxb294uxac00-5}

\subsection{막대그래프}\label{uxb9c9uxb300uxadf8uxb798uxd504-6}

\subsection{Red, Black 간에 닮았는가?}\label{red-black-uxac04uxc5d0-uxb2eeuxc558uxb294uxac00-5}

\subsection{Mosaic Plot}\label{mosaic-plot-11}

\chapter{7주차 데이터 실험 집계}\label{uxc8fcuxcc28-uxb370uxc774uxd130-uxc2e4uxd5d8-uxc9d1uxacc4-6}

Placeholder

\section{실험의 목적}\label{uxc2e4uxd5d8uxc758-uxbaa9uxc801-6}

\subsection{Red, Black을 잘못 표시한 사람들}\label{red-blackuxc744-uxc798uxbabb-uxd45cuxc2dcuxd55c-uxc0acuxb78cuxb4e4-6}

\subsection{응답인원의 Red, Black}\label{uxc751uxb2f5uxc778uxc6d0uxc758-red-black-6}

\section{Q1. 통계학의 기본원리}\label{q1.-uxd1b5uxacc4uxd559uxc758-uxae30uxbcf8uxc6d0uxb9ac}

\subsection{공평하게 추출하면 \ldots{}}\label{uxacf5uxd3c9uxd558uxac8c-uxcd94uxcd9cuxd558uxba74}

\subsection{공평하게 추출하면 \ldots{} (\%)}\label{uxacf5uxd3c9uxd558uxac8c-uxcd94uxcd9cuxd558uxba74-1}

\section{Q2. 리터러리 다이제스트의 실패}\label{q2.-uxb9acuxd130uxb7ecuxb9ac-uxb2e4uxc774uxc81cuxc2a4uxd2b8uxc758-uxc2e4uxd328}

\subsection{Selection Bias}\label{selection-bias}

\subsection{Selection Bias (\%)}\label{selection-bias-1}

\section{Q3. 1948년, 여론조사가 듀이를 당선시킨 해}\label{q3.-1948uxb144-uxc5ecuxb860uxc870uxc0acuxac00-uxb4c0uxc774uxb97c-uxb2f9uxc120uxc2dcuxd0a8-uxd574}

\subsection{할당법의 문제점}\label{uxd560uxb2f9uxbc95uxc758-uxbb38uxc81cuxc810}

\subsection{할당법의 문제점(\%)}\label{uxd560uxb2f9uxbc95uxc758-uxbb38uxc81cuxc810-1}

\section{Q4. 1948 미 대선 이후}\label{q4.-1948-uxbbf8-uxb300uxc120-uxc774uxd6c4}

\subsection{확률적 표본추출방법 도입}\label{uxd655uxb960uxc801-uxd45cuxbcf8uxcd94uxcd9cuxbc29uxbc95-uxb3c4uxc785}

\subsection{확률적 표본추출방법 도입 \ldots{} (\%)}\label{uxd655uxb960uxc801-uxd45cuxbcf8uxcd94uxcd9cuxbc29uxbc95-uxb3c4uxc785-1}

\section{Q5. 표본오차를 반으로 줄이려면?}\label{q5.-uxd45cuxbcf8uxc624uxcc28uxb97c-uxbc18uxc73cuxb85c-uxc904uxc774uxb824uxba74}

\subsection{4배로 늘려야}\label{uxbc30uxb85c-uxb298uxb824uxc57c}

\subsection{4배로 눌려야 (\%)}\label{uxbc30uxb85c-uxb20cuxb824uxc57c}

\section{Q6. 대선 여론조사의 목표모집단?}\label{q6.-uxb300uxc120-uxc5ecuxb860uxc870uxc0acuxc758-uxbaa9uxd45cuxbaa8uxc9d1uxb2e8}

\subsection{선거당일 투표하는 유권자 전체}\label{uxc120uxac70uxb2f9uxc77c-uxd22cuxd45cuxd558uxb294-uxc720uxad8cuxc790-uxc804uxccb4}

\subsection{선거당일 투표하는 유권자 전체(\%)}\label{uxc120uxac70uxb2f9uxc77c-uxd22cuxd45cuxd558uxb294-uxc720uxad8cuxc790-uxc804uxccb4-1}

\section{Wason Selection Task}\label{wason-selection-task}

\subsection{Red. Q7에 추상적 문제, Q8에 구체적 문제}\label{red.-q7uxc5d0-uxcd94uxc0c1uxc801-uxbb38uxc81c-q8uxc5d0-uxad6cuxccb4uxc801-uxbb38uxc81c}

\subsection{Black. Q7에 구체적 문제, Q8에 추상적 문제}\label{black.-q7uxc5d0-uxad6cuxccb4uxc801-uxbb38uxc81c-q8uxc5d0-uxcd94uxc0c1uxc801-uxbb38uxc81c}

\section{Q7. Red에 추상적 질문, Black에 구체적 질문}\label{q7.-reduxc5d0-uxcd94uxc0c1uxc801-uxc9c8uxbb38-blackuxc5d0-uxad6cuxccb4uxc801-uxc9c8uxbb38}

\subsection{집계}\label{uxc9d1uxacc4-14}

\subsection{\% 비교}\label{uxbe44uxad50-3}

\subsection{Mosaic Plot}\label{mosaic-plot-12}

\section{Q8. Red에 구체적 질문, Black에 추상적 질문}\label{q8.-reduxc5d0-uxad6cuxccb4uxc801-uxc9c8uxbb38-blackuxc5d0-uxcd94uxc0c1uxc801-uxc9c8uxbb38}

\subsection{집계}\label{uxc9d1uxacc4-15}

\subsection{\% 비교.}\label{uxbe44uxad50.-3}

\subsection{Mosaic Plot}\label{mosaic-plot-13}

\subsection{정답율 비교}\label{uxc815uxb2f5uxc728-uxbe44uxad50}

\section{Q9. 인지적 편향과 오류}\label{q9.-uxc778uxc9c0uxc801-uxd3b8uxd5a5uxacfc-uxc624uxb958}

\subsection{집계}\label{uxc9d1uxacc4-16}

\subsection{\% 비교.}\label{uxbe44uxad50.-4}

\subsection{Mosaic Plot}\label{mosaic-plot-14}

\section{학습 순서의 영향}\label{uxd559uxc2b5-uxc21cuxc11cuxc758-uxc601uxd5a5}

\subsection{집계표}\label{uxc9d1uxacc4uxd45c-2}

\subsection{\% 비교}\label{uxbe44uxad50-4}

\subsection{합산}\label{uxd569uxc0b0-2}

\subsection{집계표}\label{uxc9d1uxacc4uxd45c-3}

\subsection{\% 비교}\label{uxbe44uxad50-5}

\subsection{Barplot}\label{barplot-1}

\section{마감 시간으로부터 제출 시간의 분포}\label{uxb9c8uxac10-uxc2dcuxac04uxc73cuxb85cuxbd80uxd130-uxc81cuxcd9c-uxc2dcuxac04uxc758-uxbd84uxd3ec-6}

\subsection{분포표}\label{uxbd84uxd3ecuxd45c-7}

\subsection{날마다 고르게 제출하는가?}\label{uxb0a0uxb9c8uxb2e4-uxace0uxb974uxac8c-uxc81cuxcd9cuxd558uxb294uxac00-6}

\subsection{막대그래프}\label{uxb9c9uxb300uxadf8uxb798uxd504-7}

\subsection{Red, Black 간에 닮았는가?}\label{red-black-uxac04uxc5d0-uxb2eeuxc558uxb294uxac00-6}

\subsection{Mosaic Plot}\label{mosaic-plot-15}

\end{document}
